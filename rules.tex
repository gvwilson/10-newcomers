\documentclass[10pt,letterpaper]{article}
\usepackage[top=0.85in,left=2.75in,footskip=0.75in]{geometry}

% amsmath and amssymb packages, useful for mathematical formulas and symbols
\usepackage{amsmath,amssymb}

% Use adjustwidth environment to exceed column width (see example table in text)
\usepackage{changepage}

% Use Unicode characters when possible
\usepackage[utf8x]{inputenc}

% textcomp package and marvosym package for additional characters
\usepackage{textcomp,marvosym}

% cite package, to clean up citations in the main text. Do not remove.
\usepackage{cite}

% Use nameref to cite supporting information files (see Supporting Information section for more info)
\usepackage{nameref,hyperref}

% line numbers
\usepackage[right]{lineno}

% ligatures disabled
\usepackage{microtype}
\DisableLigatures[f]{encoding = *, family = * }

% color can be used to apply background shading to table cells only
\usepackage[table]{xcolor}

% array package and thick rules for tables
\usepackage{array}

% enumerate package lets us use letters instead of numbers
\usepackage{enumerate}

% create "+" rule type for thick vertical lines
\newcolumntype{+}{!{\vrule width 2pt}}

% create \thickcline for thick horizontal lines of variable length
\newlength\savedwidth
\newcommand\thickcline[1]{%
  \noalign{\global\savedwidth\arrayrulewidth\global\arrayrulewidth 2pt}%
  \cline{#1}%
  \noalign{\vskip\arrayrulewidth}%
  \noalign{\global\arrayrulewidth\savedwidth}%
}

% \thickhline command for thick horizontal lines that span the table
\newcommand\thickhline{\noalign{\global\savedwidth\arrayrulewidth\global\arrayrulewidth 2pt}%
\hline
\noalign{\global\arrayrulewidth\savedwidth}}

\usepackage{color}

% Remove comment for double spacing
%\usepackage{setspace}
%\doublespacing

% Text layout
\raggedright
\setlength{\parindent}{0.5cm}
\textwidth 5.25in
\textheight 8.75in

% Bold the 'Figure #' in the caption and separate it from the title/caption with a period
% Captions will be left justified
\usepackage[aboveskip=1pt,labelfont=bf,labelsep=period,justification=raggedright,singlelinecheck=off]{caption}
\renewcommand{\figurename}{Fig}

% Use the PLoS provided BiBTeX style
\bibliographystyle{plos2015}

% Remove brackets from numbering in List of References
\makeatletter
\renewcommand{\@biblabel}[1]{\quad#1.}
\makeatother

% Leave date blank
\date{}

% Header and Footer with logo
\usepackage{lastpage,fancyhdr,graphicx}
\usepackage{epstopdf}
\pagestyle{myheadings}
\pagestyle{fancy}
\fancyhf{}
\setlength{\headheight}{27.023pt}
\lhead{\includegraphics[width=2.0in]{PLOS-submission.eps}}
\rfoot{\thepage/\pageref{LastPage}}
\renewcommand{\footrule}{\hrule height 2pt \vspace{2mm}}
\fancyheadoffset[L]{2.25in}
\fancyfootoffset[L]{2.25in}
\lfoot{\sf PLOS}


\newcommand{\rulemajor}[1]{\section*{#1}}
\newcommand{\withurl}[2]{{#1}\footnote{{\texttt{#2}}}}

\begin{document}
\vspace*{0.2in}

\begin{flushleft}
{\Large
\textbf\newline{Ten Simple Rules for Helping Newcomers Become Contributors to Open Source Projects}
}
\newline
\\
{Dan Sholler}\textsuperscript{1{\ddag}},
{Igor Steinmacher}\textsuperscript{2{\ddag}},
{Denae Ford}\textsuperscript{3{\ddag}},
{Mara Averick}\textsuperscript{4{\ddag}},
{Mike Hoye}\textsuperscript{5{\ddag}},
{Greg Wilson}\textsuperscript{4{\ddag}*}
\\
\bigskip
\textbf{1} Berkeley Institute for Data Science / sholler.daniel@gmail.com \\
\textbf{2} Northern Arizona University/ igor.steinmacher@nau.edu \\\bigskip
\textbf{3} North Carolina State University / dford3@ncsu.edu \\
\textbf{4} RStudio, Inc. / \{mara.averick, greg.wilson\}@rstudio.com \\
\textbf{5} Mozilla Corporation / mhoye@mozilla.com \\
* Corresponding author. \\
\bigskip
{\ddag} These authors contributed equally to this work.
\end{flushleft}

\section*{Abstract}

In order to survive and thrive,
a community needs to attract and retain new members
and help them be productive.
The 10 simple rules outlined in this paper
describe ways to do this that are suitable for small and medium-sized projects,
based on current research,
and have demonstrated their effectiveness
in a wide variety of open source and open science projects.

\section*{Author Summary}

Research has always been a communal activity,
but researchers have only recently begun to pay attention to
what we know know about why some communities thrive and others fail.
Attracting contributors,
retaining them,
and helping them be more productive
have all been the subject of intensive study.
This paper presents 10 findings from those studies that community leaders can apply immediately
and explains why they work.

\section*{Introduction}

In order to survive and thrive,
a community needs to attract new members,
retain them,
and help them be productive contributors.
As openness becomes the norm in scientific research and software development,
knowing how to do this has become a necessary skill for principal investigators and other project leaders.
While every situation is unique,
A growing body of knowledge in sociology, anthropology, education, and software engineering
can guide decisions about what to do to facilitate this.

But what exactly do we mean by ``community''?
In the case of open source and open science,
the most usual meaning is a ``community of practice''.
As defined by Lave and Wenger~\cite{lave1991,wenger1999},
groups as diverse as knitting circles,
oncology researchers,
and the organizers of amateur theatrical groups
share three key characteristics:

\begin{enumerate}

\item Participants have a common purpose or product that they work on or toward.

\item They are mutually engaged, i.e., they assist and mentor each another.

\item They develop shared resources and domain knowledge.

\end{enumerate}

Brown~\cite{brown2019} specializes this to define a ``community of effort'' as,
``{\ldots}a community formed in pursuit of a common goal.
The goal can be definite or indefinite in time,
and may not be clearly defined,
but it's something that (generally speaking) the community is aligned on.''
An open source software projects like Mozilla Firefox~\cite{mozilla} is one example:
a mix of paid professionals,
highly-involved volunteers,
and occasional passers by create software,
documentation,
and tutorials
while also organizing events,
answering questions in online forums,
mentoring newcomers,
and advocating for open standards.

People working to preserve coral reefs in the face of global climate change are an example of
a more loosely connected community of effort.
There is no single central organization coordinating their work,
but the scientists who study coral reefs,
the environmentalists who work to protect them,
and the citizens who support them financially and politically
are aware of each other's efforts,
coordinate with each other in various ad hoc ways,
and are conscious of contributing toward a shared purpose.

Every community of effort has its own unique features,
but they have enough in common to profit from one another's experience.
The 10 rules laid out below are based on studies of such communities
and on the authors' experience as members,
leaders,
and observers of them.
Our focus is on small and medium-sized projects,
i.e.,
ones that have a handful to a few dozen participants
and are a few months to a few years old,
but do not (yet) have any formal legal standing
such as incorporation as a non-profit.

\rulemajor{Rule 1: Have and enforce a code of conduct.}

The first and most important rule is
to create explicit guidelines for members of the community to follow.
This helps ensure that everyone---not just newcomers---will find the environment healthy and welcoming.
Doing this also sends a clear signal that the community actually has standards:
many potential contributors will be painfully familiar with communities that don't,
and are more likely to give yours a try if they believe
it's more than a troll-infested chat room.
Explicit guidelines also make the project more socially accessible to people from differing cultural backgrounds,
as it helps them understand what expectations are and how they may differ from what they're used to.

An increasingly popular strategy for establishing guidelines is to adopt a Code of Conduct.
Empirical research on codes of conduct is still in its infancy~\cite{tourani2017},
but many projects such as rOpenSci~\cite{ropensci-coc},
NumPy~\cite{numpy-coc},
and Project Jupyter~\cite{jupyter-coc}
have adopted the Contributor Covenant~\cite{covenant}
or used other frameworks such as SciPy's Code of Conduct~\cite{scipy-coc}.
However, several questions around reporting violations and enforcing rules remain open.
What is the appropriate mechanism for reporting violations?
Who should receive reports?
What types of actions can projects take when a member of the community violates the code?
Projects may vary in how they address these questions based on characteristics of size, technology choices, or human resources
(such as the presence of a community manager).
However, some must-haves are clear based on our collective experience and code of conduct guides~\cite{aurora2019}.

For instance,
projects should designate at least one independent party
(i.e., an individual not employed by or otherwise deeply tied the project)
to receive and review reports.
An independent party offers a degree of objectivity
and can help to protect reporters from hesitating to report out of fear of retribution or damage to their reputation.
When possible,
the independent party should be part of a more extensive code of conduct committee made up of several people with varied characteristics
(e.g., gender identity, race, ethnicity, roles in the community).
Any member of the committee implicated in the incident should be recused from reviewing the violations.

Project leaders should also develop some enforcement mechanisms and,
when safe for the reporter,
publicize the enforcement decision.
Enforcing the code and announcing decisions are critical;
otherwise,
the community will quickly recognize the code as meaningless.
Committees can consider several courses of action once investigations are complete:
verbal or written warnings,
limits on access to project communication avenues (e.g., Slack channels or mailing lists),
or in severe cases,
suspension or expulsion from contributing to the project.

\rulemajor{Rule 2: Make governance and licensing explicit.}

Raymond's ``The Cathedral and the Bazaar''~\cite{raymond2001}
described an egalitarian world in which everyone could contribute equally to open projects.
Two decades later,
we can see how unequal and unwelcoming the supposedly egalitarian ``bazaar'' of open source can be
if authority lies with those willing to shout loudest and longest.
As Bezroukov pointed out~\cite{bezroukov1999},
Raymond ignored the realities of how power arises,
becomes concentrated in a few hands,
and is then used to perpetuate itself.

Bezroukov's criticism drew on Freeman's influential essay ``The Tyranny of Structurelessness''~\cite{freeman1972},
which explained how an apparent lack of structure in organizations ``{\ldots}too often disguised an informal,
unacknowledged and unaccountable leadership that was all the more pernicious because its very existence was denied.''
The solution is to make a project's governance explicit
so that people know who makes which decisions.

Large, well-established projects that incorporate as non-profits are required to promulgate bylaws,
such as those for the Python Software Foundation~\cite{psf-bylaws}.
What smaller projects should do is less well documented,
but the ``Social and Political Infrastructure'' chapter of~\cite{fogel2005}
describes two models:

\begin{itemize}
	
\item
  A ``benevolent dictator'' (who might better be called a ``community-approved arbitrator'') has final say,
  but their principal responsibility is to manage conversation to achieve consensus.
	
\item
  Elected representation and/or community votes on key issues.
  In the Carpentries,
  for example,
  anyone who satisfies any of the conditions below is considered a voting member of the organization~\cite{carpentries-bylaws}:
  \begin{enumerate}
  \item
    everyone who has completed instructor certification in the preceding year;
  \item
    everyone who has completed certification in the last two years and taught at least one workshop;
  \item
    everyone who has been certified for more than two years and has taught at least twice in that time; and
  \item
    anyone who has made a significant contribution to lesson development, infrastructure, or other activities
    as determined by the Executive Council.
  \end{enumerate}

\end{itemize}

More complex models are possible~\cite{apache-governance},
but the most important thing is to decide on the rules well in advance of contentious issues emerging,
since tempers may already be running hot by the time this point is reached.

As well as making governance explicit,
it is essential that open projects be clear about licensing.
Who can use the data collected by the software for what purposes,
and what attribution do they have to give?
How must they acknowledge the project and/or its contributors?
Who holds the copyright on contributed material?
Most projects now place this information in files called \texttt{LICENSE.md} and \texttt{CITATION.md}
to make them easy to find.
Wherever they are put,
the most important thing is to make it clear before people begin contributing.

\rulemajor{Rule 3: Be welcoming.}

As Fogel said~\cite{fogel2005},
``If a project doesn't make a good first impression, newcomers may wait a long time before giving it a second chance.''
Other authors have empirically confirmed the importance of kind and polite social environments
in open source projects \cite{singh2012,steinmacher2013,steinmacher2018a}.
Therefore,
projects should not just say that they welcome new members:
they should make a proactive effort to foster positive feelings in them.

\begin{itemize}

\item
  Post a welcome message on the project's social media pages, Slack channels, forums, or email lists.
  Projects might consider maintaining a dedicated ``Welcome'' channel or list,
  where a project lead or community manager writes a short post asking newcomers to introduce themselves.

\item
  Offer assistance in finding ways to make an initial contribution.

\item
  Direct the newcomer to project members who have a similar background or skill set
  so as to demonstrate fit to the newcomer.

\item
  Point the newcomer to essential project resources (e.g., the contribution guidelines).

\item
  Clearly identify work items they can start with.
  A growing number of projects explicitly tag bugs or issues as ``suitable for newcomers'',
  and ask established members not to fix them
  in order to ensure there are suitable places for new arrivals to start work.
  
\end{itemize}

Projects can further designate one or two members to serve as a point-of-contact for each newcomer.
Doing this may reduce the newcomer's hesitancy to ask questions,
particularly when they are told from the outset that there are no dumb questions in the community.

\rulemajor{Rule 4: Help potential contributors evaluate if the project is a good fit.}

People could contribute to many different projects;
the first and most important step in being welcoming is to help them determine whether
your project is a good fit for their interests and abilities.
Their decision to contribute can be related to reputation or external needs,
but also to a desire to learn or give back to the community.
In all of these cases,
the more you help newcomers understand if this is the right project for them,
the more quickly they will either start contributing or look elsewhere.

To do this,
the project should explicitly state what the different types of skills required are.
This information should be easily accessible and guide new members to the tasks they may handle.
LibreOffice,
for example,
provides a way for developers to filter available tasks by required skills and difficulty~\cite{libreoffice-filtered}.

The project should also help developers evaluate their skills,
since ``basic Python skills'' means very different things to different people.
Tools like My GitHub Resume~\cite{my-github-resume} and Visual Resume~\cite{sarma2016}
that aggregate information from previous contributions can help with this assessment,
while the OpenHatch project~\cite{openhatch} aggregated entry-level issues from a variety of open source projects
and classified them according to language and other required skills
to provide a one-stop portal for finding appropriate projects.

\rulemajor{Rule 5: Develop forms of legitimate peripheral participation}

A core concept in the theory of communities of practice is that of
legitimate peripheral participation (LPP)~\cite{lave1991,wenger1999}.
Newcomers become members of a community by participating in simple, low-risk tasks
that further the goals of the community.
Through these peripheral activities,
newcomers become acquainted with the community's tasks, vocabulary, and governance
so that they can ease into the project.

In communities such as GitHub,
core activities such as committing code and submitting pull requests can be socially daunting for newcomers~\cite{steinmacher2015}.
One way to encourage LPP in this case is to encourage newcomers to add an issue to a repository when they notice a bug
or to join the dialog on a recently submitted pull request.
Another way is to have newcomers help with documentation,
particularly with translation and localization,
and a third (mentioned in Rule~3) is to mark some issues as suitable for newcomers.

Building multiple ways of participating in a community demonstrates the variety of approaches newcomers can take to join the community.
This further demonstrates that there is not just one way to make technical contributions.
For example,
the main form of interaction in the community on Stack Overflow is to ask a question and posting an answer,
but engaging in that type of interaction can present barriers some users
including an intimidating community size and fear of negative feedback~\cite{ford2016}.
Thus, it is important to provide additional forms of participation.
On Stack Overflow, this is demonstrated through the ability to edit questions and answer without the restriction of reputation points.
Developing a pathway to participation can decrease the presence of barriers.
In studying the evolution of how content is formed in these communities~\cite{baltes2018},
newcomers can better understand the norms of a community and the best way to contribute~\cite{ford2018}.

\rulemajor{Rule 6: Make knowledge findable and keep it up to date.}

When starting to contribute to a project,
newcomers need to learn their way around it.
However,
newcomers to software projects are like explorers who must orient themselves within an unfamiliar landscape~\cite{dagenais2010}.
Thus,
it is important to make sure that all necessary information is accessible
(or at least searchable)
by newcomers during their first steps.

Information that is spread out usually makes newcomers feel lost and disoriented.
Given the different possibilities of places to maintain information
(e.g., wikis, files in the repository, shared documents, old tweets or Slack messages, and email archives),
it is important to keep information about a specific topic consolidated in a single place
so that newcomers do not need to navigate multiple data sources to find what they need.
The literature has shown that organizing the information make newcomers more confident and oriented~\cite{steinmacher2016}.
This organization is something important for technical and process documentation but also for communication channels.
Another suggestion is, therefore, to stick to a small number of communication channels
and to clearly define the goals for each of them.

At the same time,
outdated documentation may lead newcomers to a wrong understanding of the project,
which is also demotivating.
While it may be hard to keep documentation up-to-date,
community members should remove outdated information or, at least, clearly identify it as outdated.
Making newcomers aware of the absence or the status of a document can save their time and set their expectations.
By recognizing the obsolescence of the information,
communities may request help from the newcomers as a way to foster their contribution.

\rulemajor{Rule 7: Provide an easy, complete, and up-to-date guide to contributing.}

A project can guide newcomers toward desired practices
by providing them with ``how to contribute'' guidelines in easy-to-find, readily-available places.
Many projects follow GitHub's recommendation for placing such information in a \texttt{CONTRIBUTING.md} file~\cite{github-rec}.
Other projects,
such as the Apache Open Office Suite and rOpenSci,
provide newcomer manuals and learning modules accessed through a web interface~\cite{apache-guidelines,ropensci-guidelines}.
Still others take a more interactive approach;
for example,
the GNOME project's Newcomers Guide~\cite{gnome-newcomers}, walks potential contributors through the contribution pipeline:
choosing a project,
acquiring and installing the necessary computing tools,
finding problems or choosing issues to work on,
submitting changes,
and following up on feedback.

No matter how they are presented,
such guidelines do more than just describe how to contribute
and help newcomers feel they are on equal footing with veteran project members.
First,
their mere existence can ease newcomers' hesitation
about whether or not their work is sufficient and suitable for the project.
Second,
they provide a centralized, well-organized description of resources
that a newcomer can consult while learning to navigate the project's technical and social environments~\cite{zanatta2017}.
Guidelines also acclimate newcomers to the norms of work and communication,
particularly when items such as necessary computing tools and codes of conduct are foregrounded.

Finally,
such guidelines should be clear about how contributions will be acknowledged
and how they can be used.
A pointer to the license and citation documents (Rule~2) is a start,
but a summary in plain language will be more helpful for newcomers.

Project decision-makers should keep in mind
that their perceptions of what constitutes clear, easy-to-find contribution guidelines
may not align with the perceptions of the newcomers themselves.
Continual reevaluation of the guidelines based on community member feedback is essential
to ensuring that contribution guidelines are achieving desired effects
and do not fall out of date (e.g., as technical and social elements of the community change).
Contributing to the contribution guidelines is therefore
an excellent opportunity for legitimate peripheral participation.

\rulemajor{Rule 8: Use opportunities for in-person interaction, but with care.}

Open source software projects often rely heavily on remote workers communicating via text, audio, and video.
Research on face-to-face and audio/video-mediated communication is mixed
with regard to their comparative effectiveness~\cite{doherty1997,gallupe1990,nardi2002},
but demonstrates that each form has benefits and drawbacks.
In-person interaction is valuable for uninterrupted, synchronous dialog
and helps to establish mutual understanding in a streamlined way~\cite{omalley1996}.
Projects can therefore benefit from engaging newcomers in in-person interaction from time to time.

According to Huppenkothen and colleagues~\cite{huppenkothen2018},
newcomers may particularly benefit from hackweeks that,
``{\ldots}combine structured periods focused on pedagogy
(often with an emphasis on statistical and computational techniques)
and less structured periods devoted to hacks and creative projects,
with the goal of encouraging collaboration and learning among people at various stages of their career.''
Building newcomer-friendly events and activities into a hackweek
might serve as an effective approach for acclimating newcomers to the project and its community
as well as highlight the potential avenues for newcomer contributions.

However,
projects should exercise caution when asking newcomers to communicate in person.
First, potential contributors might shy away from the project if they are introverted,
suffer from social anxiety,
or have had bad experiences in the past in face-to-face settings.
Establishing and publicizing a Code of Conduct helps allay these concerns,
but some newcomers may still feel uncomfortable in group settings.
In this case,
not going to a meetup may leave them feeling less a part of the community.

Face-to-face communication also involves forms of information exchange
that are not easily captured and archived for all project members to see.
For example,
collocated project members might hash out ideas on whiteboards,
by scribbling notes,
or through informal chats.
Even when transcribing and/or taking photos of these is possible,
important contextual information may be lost~\cite{cherubini2007}.
Decisions and changes may seem to come out of nowhere when evaluated by a non-attendee,
so project leads should develop universally-accessible ways to communicate and explain the results of in-person activities.

\rulemajor{Rule 9: Make it easy for newcomers to set up.}

The experience of getting set up to work on a project---the moment they go from ``I want to help''
to ``I'm able to help'' to ``I'm helping''---is often a contributor's first impressions of
the experience of participating in that community.
This makes the developer setup process a critical moment in your relationship with every new contributor,
which in turn means that any complexity or confusion at this point is a significant barrier to participation~\cite{steinmacher2014}.
By treating the process of getting involved with the same care and attention you give to the product itself,
you're making it clear that you value those contributors' time and effort,
and forestalling reactions like this~\cite{steinmacher2018b}:

\begin{quote}
  I am still trying to build, because many errors occurred{\ldots}
  I was expecting to move forward,
  because so far I did not have time to look at the source code{\ldots}
  It is frustrating.
\end{quote}

This work does not just benefit newcomers:
it also helps retention of existing intermittent contributors and the same work that makes your project more
accessible to new contributors today will do the same for future you.
Wheelchair ramps and the buttons that open heavy doors are not just used by those in wheelchairs:
they are just as helpful to people with strollers or one too many bags of groceries.
None of us are ever more than a sprained ankle away from desperately wanting that wheelchair ramp to be there.
In that same vein, a drive failure will someday force you to download a gigabyte of data
and reinstall some software, inevitably at the least convenient moment imaginable.
There is therefore a lot to be gained from automating as much of your setup process you can
and thoroughly documenting whatever you can't.

\rulemajor{Rule 10: Follow up on success.}

People in open source sometimes joke that
a programmer is someone who will do something for a laptop sticker
that they wouldn't do for a hundred dollars.
This joke acknowledges that gratitude and recognition are the most powerful tools available for community builders and maintainers.
Our last rule is therefore as important as our first:
acknowledge newcomers' contributions and thank them for their work.
Every hour that someone has given your project may be an hour taken away from their personal life
or their official employment;
recognize that fact
and make it clear that while more hours would be welcome,
you do not expect them to make unsustainable sacrifices.

This is also a good time to figure out where and how the newcomer might help in the longer term.
Once they have carried their first contribution over the line,
you and they are likely to have a better sense of what they have to offer
and how the project can help them.
Helping them find the next problem they might want to work on
or pointing them at the next thing they might enjoy reading
is both helpful and supportive.
In particular,
encouraging them to help the next wave of newcomers
is both a good way to recognize what they have learned,
and an effective way to pass it on.

Mentoring programs are a popular way to do this.
However,
their effectiveness appears mixed.
\cite{fagerholm2014} found that,
``{\ldots}developers receiving deliberate onboarding support through mentoring
were more active at an earlier stage than developers entering projects through conventional means.''
In contrast,
\cite{labuschagne2015} found that,
``{\ldots}developers who join an organization through these programs
are half as likely to transition into long-term community members
than developers who do not use these programs{\ldots}
although developers who do succeed through these programs find them valuable.''

One explanation for this disparity is that people become members of open projects for different reasons,
and hence respond to things like mentoring programs in different ways.
For example,
Barcomb et al.\ identified four types of episodic or intermittent contributors to open source projects~\cite{barcomb2019},
while M\"{a}enp\"{a}\"{a} et al.\ looked at how to reconcile
the competing yet complementary needs of stakeholders in hybrid open/commercial projects.
More research is needed,
but as openness becomes the norm in research,
doing it well becomes a core skill for every researcher.

\bibliography{rules}

\end{document}
