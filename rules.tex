\documentclass[10pt,letterpaper]{article}
\usepackage[top=0.85in,left=2.75in,footskip=0.75in]{geometry}

% amsmath and amssymb packages, useful for mathematical formulas and symbols
\usepackage{amsmath,amssymb}

% Use adjustwidth environment to exceed column width (see example table in text)
\usepackage{changepage}

% Use Unicode characters when possible
\usepackage[utf8x]{inputenc}

% textcomp package and marvosym package for additional characters
\usepackage{textcomp,marvosym}

% cite package, to clean up citations in the main text. Do not remove.
\usepackage{cite}

% Use nameref to cite supporting information files (see Supporting Information section for more info)
\usepackage{nameref,hyperref}

% line numbers
\usepackage[right]{lineno}

% ligatures disabled
\usepackage{microtype}
\DisableLigatures[f]{encoding = *, family = * }

% color can be used to apply background shading to table cells only
\usepackage[table]{xcolor}

% array package and thick rules for tables
\usepackage{array}

% enumerate package lets us use letters instead of numbers
\usepackage{enumerate}

% create "+" rule type for thick vertical lines
\newcolumntype{+}{!{\vrule width 2pt}}

% create \thickcline for thick horizontal lines of variable length
\newlength\savedwidth
\newcommand\thickcline[1]{%
  \noalign{\global\savedwidth\arrayrulewidth\global\arrayrulewidth 2pt}%
  \cline{#1}%
  \noalign{\vskip\arrayrulewidth}%
  \noalign{\global\arrayrulewidth\savedwidth}%
}

% \thickhline command for thick horizontal lines that span the table
\newcommand\thickhline{\noalign{\global\savedwidth\arrayrulewidth\global\arrayrulewidth 2pt}%
\hline
\noalign{\global\arrayrulewidth\savedwidth}}

\usepackage{color}

% Remove comment for double spacing
%\usepackage{setspace}
%\doublespacing

% Text layout
\raggedright
\setlength{\parindent}{0.5cm}
\textwidth 5.25in
\textheight 8.75in

% Bold the 'Figure #' in the caption and separate it from the title/caption with a period
% Captions will be left justified
\usepackage[aboveskip=1pt,labelfont=bf,labelsep=period,justification=raggedright,singlelinecheck=off]{caption}
\renewcommand{\figurename}{Fig}

% Use the PLoS provided BiBTeX style
\bibliographystyle{plos2015}

% Remove brackets from numbering in List of References
\makeatletter
\renewcommand{\@biblabel}[1]{\quad#1.}
\makeatother

% Leave date blank
\date{}

% Header and Footer with logo
\usepackage{lastpage,fancyhdr,graphicx}
\usepackage{epstopdf}
\pagestyle{myheadings}
\pagestyle{fancy}
\fancyhf{}
\setlength{\headheight}{27.023pt}
\lhead{\includegraphics[width=2.0in]{PLOS-submission.eps}}
\rfoot{\thepage/\pageref{LastPage}}
\renewcommand{\footrule}{\hrule height 2pt \vspace{2mm}}
\fancyheadoffset[L]{2.25in}
\fancyfootoffset[L]{2.25in}
\lfoot{\sf PLOS}


%% Define per-paper macros.
\newcommand{\withurl}[2]{{#1}\footnote{\texttt{#2}}}
\newcommand{\rulemajor}[1]{\section{#1}}
\begin{document}
\vspace*{0.2in}

\begin{flushleft}
{\Large
\textbf\newline{Ten Simple Rules for Helping Newcomers Become Contributors to Open Source Projects}
}
\newline
\\
{Mara Averick}\textsuperscript{1{\ddag}},
{Denae Ford}\textsuperscript{2{\ddag}},
{Mike Hoye}\textsuperscript{3{\ddag}},
{Jesse Mostipak}\textsuperscript{4{\ddag}},
{Dan Sholler}\textsuperscript{5{\ddag}},
{Greg~Wilson}\textsuperscript{1{\ddag}*}
\\
\bigskip
\textbf{1} RStudio, Inc. / \{mara.averick, greg.wilson\}@rstudio.com \\
\textbf{2} AFFILIATION / EMAIL \\
\textbf{3} AFFILIATION / EMAIL \\
\textbf{4} AFFILIATION / EMAIL \\
\bigskip
* Corresponding author. \\
\bigskip
{\ddag} These authors contributed equally to this work.
\end{flushleft}

\section*{Abstract}

FIXME: abstract.

\section*{Author Summary}

FIXME: one-paragraph summary.

\section*{Introduction}

According to \cite{b:wenger-cop}, communities of practice have three key characteristics:

\begin{enumerate}

\item participants have a common purpose or product that they work on or toward;

\item they are mutually engaged, i.e., they assist and mentor each another; and

\item they develop shared resources and domain knowledge.

\end{enumerate}

\rulemajor{Rule 1}

\rulemajor{Rule 2}

\rulemajor{Rule 3}

\rulemajor{Rule 4}

\rulemajor{Rule 5}

\rulemajor{Rule 6}

\rulemajor{Rule 7}

\rulemajor{Rule 8}

\rulemajor{Rule 9}

\rulemajor{Rule 10}

\section*{Conclusion}

\bibliographystyle{plos2015}
\bibliography{rules}

\end{document}
